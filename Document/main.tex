%% This is file `DEMO-TUDaThesis.tex' version 3.32 (2023/06/19),
%% it is part of
%% TUDa-CI -- Corporate Design for TU Darmstadt
%% ----------------------------------------------------------------------------
%%
%%  Copyright (C) 2018--2023 by Marei Peischl <marei@peitex.de>
%%
%% ============================================================================
%% This work may be distributed and/or modified under the
%% conditions of the LaTeX Project Public License, either version 1.3c
%% of this license or (at your option) any later version.
%% The latest version of this license is in
%% http://www.latex-project.org/lppl.txt
%% and version 1.3c or later is part of all distributions of LaTeX
%% version 2008/05/04 or later.
%%
%% This work has the LPPL maintenance status `maintained'.
%%
%% The Current Maintainers of this work are
%%   Marei Peischl <tuda-ci@peitex.de>
%%   Markus Lazanowski <latex@ce.tu-darmstadt.de>
%%
%% The development respository can be found at
%% https://github.com/tudace/tuda_latex_templates
%% Please use the issue tracker for feedback!
%%
%% If you need a compiled version of this document, have a look at
%% http://mirror.ctan.org/macros/latex/contrib/tuda-ci/doc
%% or at the documentation directory of this package (if installed)
%% <path to your LaTeX distribution>/doc/latex/tuda-ci
%% ============================================================================
%%
% !TeX program = lualatex
%%

\documentclass[
	ngerman,
	ruledheaders=section,%Ebene bis zu der die Überschriften mit Linien abgetrennt werden, vgl. DEMO-TUDaPub
	class=report,% Basisdokumentenklasse. Wählt die Korrespondierende KOMA-Script Klasse
	thesis={type=master},% Dokumententyp Thesis, für Dissertationen siehe die Demo-Datei DEMO-TUDaPhd
	accentcolor=9c,% Auswahl der Akzentfarbe
	custommargins=true,% Ränder werden mithilfe von typearea automatisch berechnet
	marginpar=false,% Kopfzeile und Fußzeile erstrecken sich nicht über die Randnotizspalte
	%BCOR=5mm,%Bindekorrektur, falls notwendig
	parskip=half-,%Absatzkennzeichnung durch Abstand vgl. KOMA-Script
	fontsize=11pt,%Basisschriftgröße laut Corporate Design ist mit 9pt häufig zu klein
%	logofile=example-image, %Falls die Logo Dateien nicht vorliegen
]{tudapub}

% Meine Packete
\usepackage{syntax}
\renewcommand{\syntleft}{}
\renewcommand{\syntright}{}
\usepackage{listings}
\def\lstlanguagefiles{lstlean.tex}

\usepackage{color}
\definecolor{keywordcolor}{rgb}{0.7, 0.1, 0.1}   % red
\definecolor{tacticcolor}{rgb}{0.0, 0.1, 0.6}    % blue
\definecolor{commentcolor}{rgb}{0.4, 0.4, 0.4}   % grey
\definecolor{symbolcolor}{rgb}{0.0, 0.1, 0.6}    % blue
\definecolor{sortcolor}{rgb}{0.1, 0.5, 0.1}      % green
\definecolor{attributecolor}{rgb}{0.7, 0.1, 0.1} % red

% Der folgende Block ist nur bei pdfTeX auf Versionen vor April 2018 notwendig
\usepackage{iftex}
\ifPDFTeX
	\usepackage[utf8]{inputenc}%kompatibilität mit TeX Versionen vor April 2018
\fi

%%%%%%%%%%%%%%%%%%%
%Sprachanpassung & Verbesserte Trennregeln
%%%%%%%%%%%%%%%%%%%
\usepackage[english, main=ngerman]{babel}
\usepackage[autostyle]{csquotes}% Anführungszeichen vereinfacht

% Falls mit pdflatex kompiliert wird, wird microtype automatisch geladen, in diesem Fall muss diese Zeile entfernt werden, und falls weiter Optionen hinzugefügt werden sollen, muss dies über
% \PassOptionsToPackage{Optionen}{microtype}
% vor \documentclass hinzugefügt werden.
\usepackage{microtype}

%%%%%%%%%%%%%%%%%%%
%Literaturverzeichnis
%%%%%%%%%%%%%%%%%%%
\usepackage{biblatex}   % Literaturverzeichnis
\bibliography{DEMO-TUDaBibliography}


%%%%%%%%%%%%%%%%%%%
%Paketvorschläge Tabellen
%%%%%%%%%%%%%%%%%%%
%\usepackage{array}     % Basispaket für Tabellenkonfiguration, wird von den folgenden automatisch geladen
\usepackage{tabularx}   % Tabellen, die sich automatisch der Breite anpassen
%\usepackage{longtable} % Mehrseitige Tabellen
%\usepackage{xltabular} % Mehrseitige Tabellen mit anpassbarer Breite
\usepackage{booktabs}   % Verbesserte Möglichkeiten für Tabellenlayout über horizontale Linien

%%%%%%%%%%%%%%%%%%%
%Paketvorschläge Mathematik
%%%%%%%%%%%%%%%%%%%
%\usepackage{mathtools} % erweiterte Fassung von amsmath
%\usepackage{amssymb}   % erweiterter Zeichensatz
%\usepackage{siunitx}   % Einheiten

%Formatierungen für Beispiele in diesem Dokument. Im Allgemeinen nicht notwendig!
\let\file\texttt
\let\code\texttt
\let\tbs\textbackslash
\let\pck\textsf
\let\cls\textsf

\usepackage{pifont}% Zapf-Dingbats Symbole
\newcommand*{\FeatureTrue}{\ding{52}}
\newcommand*{\FeatureFalse}{\ding{56}}

\begin{document}

\Metadata{
	title=Master Thesis WiSe2023,
	author=Simon Daniel
}

\title{Distributed Programming -- Funny things in Lean with session Types}
\subtitle{TODO Untertitel}
\author[S. Daniel]{Simon Daniel}%optionales Argument ist die Signatur,
\birthplace{Worms}%Geburtsort, bei Dissertationen zwingend notwendig
\reviewer{ 1. Prof. Dr.-Ing. Mira Mezini  \and 2. David Richter M.Sc.}%Gutachter

%Diese Felder werden untereinander auf der Titelseite platziert.
%\department ist eine notwendige Angabe, siehe auch dem Abschnitt `Abweichung von den Vorgaben für die Titelseite'
\department{inf} % Das Kürzel wird automatisch ersetzt und als Studienfach gewählt, siehe Liste der Kürzel im Dokument.
\group{Software Technology Group}

\submissiondate{\today}
\examdate{\today}

% Hinweis zur Lizenz:
% TUDa-CI verwendet momentan die Lizenz CC BY-NC-ND 2.0 DE als Voreinstellung.
% Die TU Darmstadt hat jedoch die Empfehlung von dieser auf die liberalere
% CC BY 4.0 geändert. Diese erlaubt eine Verwendung bearbeiteter Versionen und
% die kommerzielle Nutzung.
% TUDa-CI wird im nächsten größeren Release ebenfalls diese Anpassung vornehmen.
% Aus diesem Grund wird empfohlen die Lizenz manuell auszuwählen.
%\tuprints{urn=XXXXX,printid=XXXX,year=2022,license=cc-by-4.0}
% To see further information on the license option in English, remove the license= key and pay attention to the warning & help message.

% \dedication{Für alle, die \TeX{} nutzen.}

\maketitle

\affidavit
% Es gibt mit Version 3.20 die Möglichkeit ein Bild als Signatur einzubinden.
% TUDa-CI kann nicht garantieren, dass dies zulässig ist oder eine eigenhändige Unterschrift ersetzt.
% Dies ist durch Studierende vor der Verwendung abzuklären.
% Die Verwendung funktioniert so:
%\affidavit[signature-image={\includegraphics[width=\width,height=1cm]{example-image}}, <hier können andere Optionen zusätzlich stehen>]

\tableofcontents


\chapter {literature}
\section{Dictionary}

(clarify the use of Agent, nodes, locations and be consistent throughout the thesis.)
Literature uses different kinds of naming conventions for talking about communication paricipants, sometimes to express a slightly different intent that share semantically the same properties. I will stick to location throughout my work, in other work you might read terms like nodes, parties, Agents. The paper also presents several extensions (... maybe add whats important)

Lamda Calculus and Pi Calculus
dont need seperate sections i think

\section{Sessions and Session Types:
an Overview}
Session Types take an important role for Choreos and describe the interaction of two (binary) or more (multi party) locations at communication level. We differentiate between \textbf{Global Session Types} that describe the sequence of directed message exchange for a session, and \textbf{Local Session Types}, which reflect the same Protocol but from one locations's perspective. The projection from a global session type to a local ST is called \textbf{End Point Projection} and basicallay translates all sending operations to send operations if location being projected to is sending, or receive operations if is receiving.

\section{MultiParty Session Types}


Essential Session Type Syntax introduced by K. Honda in the 1990

\begin{grammar}

<T> ::= end
	\alt !S.S' -- Send
	\alt ?S.S' -- Receive

\end{grammar}

\begin{grammar}
<G> ::= end
	\alt \mu t. G
	\alt t
	\alt p \rightarrow q: $\{l_i(S_i).G_i\}_{i \in I}$

\end{grammar}


Syntax of Introduction to MPST

\begin{grammar}

<S> ::= nat
	\alt int
	\alt bool

\end{grammar}

\begin{grammar}

<T> ::= end
	\alt $ \&_{i \in I} p?l_i(S_i).T_i$
	\alt $\oplus _{i \in I} q!l_i(S_i).T_i$
	\alt \mu t.T
	\alt t

\end{grammar}

Syntax of Mira / TU Darmstadt
extends the previous Global Type Syntax by the following:
\begin{grammar}

<G> ::= [b]G
	\alt G|G
	\alt G+G
	\alt \prod x:I.G
	\alt G e

\end{grammar}

Extension to Local Types
\begin{grammar}

<T> ::= [b]T
	\alt T+T
	\alt \prod x:I.T
	\alt T e

\end{grammar}

Global Type Implementation in Lean4

\begin{lstlisting}[language=lean]
inductive P where
  | IF 			: located Exp -> P -> P -> P
  | SEND_RECV   : located Exp -> located Variable -> P -> P
  | COMPUTE (v: Variable) (e: Exp) (a: Location) :   P -> P
  | END     	: Exp -> Location -> P
\end{lstlisting}

\section{HasChor: Functional Choreographic Programming for All
(Functional Pearl)}
this paper is about Choreographic programming and an implementation for the Haskell Language. Choreographies are implemented as a Library for use with Haskell and expressed as Computational monads. Might be helpful for own Implementations in Lean to look at their Design Choices for Language Design. Special remarks are on the \textbf{higher order} capabilities, meaning Choreographies can take other Choreos as a parameter as some kind of sub-protocol, as well as \textbf{location polymorphism}. Location polymorphism here means the runtime substitution of locations.

\section{Modular Compilation for Higher-Order Functional
Choreographies}
Higher order Choreographies make it non trivial to check which locations are involved in the Choreography. For example if Choreo C depends on Choreo K, the participants of C are only known after K is instantiated. This poses problems for static checks and the EPP. The paper proposes approaches to extend the lambda calcus to achieve a modular applications of an EPP (did not fully understand alot)

\section{On the Monitorability of Session Types, in Theory
and Practice}
about runtime checking and monitoring of communication but very long

\section{Multiparty Languages: The Choreographic and
Multitier Cases}

this work presents the two programming paradigms Choreographic programming and Multitier. Multitier being programm descriptions that specify the location of operations instead of explicit communication. While both approaches have different roots and went through different development, mainly because both workgroups rarely mix or push collective results, there still is a fundamental linkage between both. This similiaritys could lead both camps to \textbf{cross-fertilisation} that benefits both. This statement is backed by the comparison of two stripped down versions of an Choreo PL and a Multitier one.

\section{Certified Automatic Repair of Uncompilable Protocols}
There are Choreos that are not (directly) projectable to local Programs / types. Branching adds the challenges for the session, since continuation of a location protocal might depend on some choice that leads to different executions of two or more sub Protocols. This issue can be fixed by informing other Party members of branching choices that a location might does, called \textbf{knowledge of choice}. The paper also describes more clever ways of propagating choices, since not all choices are relevant to all locations and choices can be transitively propageted for example. The Automated Process is called \textbf{Amendment} or \textbf{repair}.


\chapter{Proposal}
Parallel and distributed systems have been established for decades by now and the relevance of communication in the field of programming has steadily increased ever since. While such Software Systems could be written with traditional PL's and paradigms there arises a whole ne set of programming error possibilities. For example deadlocks, or sending unexpected types of data to a receiver.
To make programming DS more intuively and less error prone there has been high effort in creating specialized languages. those 'new' languages usally follow one of two paradigms.
a Choreography or a Multiparty view. A Choreography tries to describes the communication from some kind of global view like a play or a dance with a construct of a directed send that specifies both sender and receiver. Multiier programming on the other hand lets the programmer shift the location of computation in code and thus masking the required communication. Still it has been shown that both approaches share strong similiaritys (Multiparty Languages: The Choreographic and
Multitier Cases).
Session Types, first introduced 1998 (i guess earlier?) and standardized by W3C in 2002, are an handy tool that has been used to statically check multiparty programs. They describe the communication layout of two or more partys and can already make strong safety guarantees by cheap static analysis. There has been many new research regarding session Types since (2015 i guess nochmal nachsehen) that focused on extending session types and their expressiveness, \par but only a few cover session types in asychronous and unreliable environments. Fault tolrence in under these circumstances can be important, so i will investigate on the special challenges that arise here. Accompanying i implement a simple Choreography Model with session type projection in Lean4 that demonstrates some of my points

\section{Research Questions}

\begin{itemize}
  \item unreliable nodes might be interesing? fault-tolerant and asynchronous systems were not covered by HasChor paper ``and other functional Choreographic languages'' do nothing to adress these difficulties (also mentioned in Modular Compilation for Higher-Order Functional Choreographies).
  \item how can session times still yield safety guarantees in asynchronous and/or unreliable systems (Multiparty Asynchronous Session Types Honda.)
  \item make Choreographys without obvious EPP projectable (like in Now it Compiles!). Are there other interesting problems than knowledge of choice?
  \item might be interesting to look into runtime verification with session types. (havent red alot about this, more in the future work / outlook sections)
\end{itemize}

\chapter{Networking in Lean4}
Lean4's standard Library provides IO operations for exchanging values between multiple Threads. While this would suffice to test the functionality of running Choreographies on a single machine, it is much more desirable to be able to split the Choreography into actual seperate programs that may run on different machines. To accomplish this I decided to implement message exchanging by Sockets.
While some PLs like Java provide a Language embedded implementation of Sockets, or other well established Languages like Haskell provide mature Librarays for a Socket API, Lean has a much younger ecosystem that seems less feature rich. The Solution i chose is the relatively young Lean4 library 'sockets for Lean 4' by Henrik Böving (TODO in Quellen reinpacken) which is hosted at https://github.com/hargoniX/socket.lean under the MIT License. It servers as a wrapper around the platform specific 'winsock.h' Windows, and the 'sys/socket.h' Unix C-librarys while providing a clean Lean interface.
\par
To abstract away the Socket setup code in my local programs i decided to reopen a new socket for every communication operation in the Choreography for simplicity

\begin{lstlisting}[language=lean]
def address.send (a: address) (msg: t) [Serialize t]: IO Unit := do
  let bytes := Serialize.to_bytes msg
  let sock ← Socket.mk .inet .stream
  sock.connect a
  let sz ← sock.send bytes
  sock.close
\end{lstlisting}

similiarly to the sock.connect operation, the 'receive' function will bind an listen to a socket with the corresponding addres. For convenience a 'broadcast' function is also available, that takes an 'List address' instead of a single address and sends msg to all addresses in the List.
\newline
Serialize is a simple TypeClass that provides the following two functions:
\begin{lstlisting}[language=lean]
class Serialize (a: Type) extends ToString a where
  to_bytes: a -> ByteArray
  from_bytes: ByteArray -> Except String a
\end{lstlisting}
By implementing this TypeClass a Type can converted to, and from a ByteArray, enabling it to be sent over the network.



\printbibliography

\end{document}
