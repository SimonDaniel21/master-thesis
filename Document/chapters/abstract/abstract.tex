\chapter*{Abstract}
\label{ch:abstract}
Parallel and distributed systems have been established for decades by now and the relevance of communication in the field of programming has steadily increased ever since. While such Software Systems could be written with traditional PL's and paradigms there arises a whole ne set of programming error possibilities. For example deadlocks, or sending unexpected types of data to a receiver.
To make programming DS more intuively and less error prone there has been high effort in creating specialized languages. those 'new' languages usally follow one of two paradigms.
a Choreography or a Multiparty view. A Choreography tries to describes the communication from some kind of global view like a play or a dance with a construct of a directed send that specifies both sender and receiver. Multiier programming on the other hand lets the programmer shift the location of computation in code and thus masking the required communication. Still it has been shown that both approaches share strong similiaritys (Multiparty Languages: The Choreographic and
Multitier Cases).
Session Types, first introduced 1998 (i guess earlier?) and standardized by W3C in 2002, are an handy tool that has been used to statically check multiparty programs. They describe the communication layout of two or more partys and can already make strong safety guarantees by cheap static analysis. There has been many new research regarding session Types since (2015 i guess nochmal nachsehen) that focused on extending session types and their expressiveness, \par but only a few cover session types in asychronous and unreliable environments. Fault tolrence in under these circumstances can be important, so i will investigate on the special challenges that arise here. Accompanying i implement a simple Choreography Model with session type projection in Lean4 that demonstrates some of my points
