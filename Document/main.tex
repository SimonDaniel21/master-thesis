%% This is file `DEMO-TUDaThesis.tex' version 3.32 (2023/06/19),
%% it is part of
%% TUDa-CI -- Corporate Design for TU Darmstadt
%% ----------------------------------------------------------------------------
%%
%%  Copyright (C) 2018--2023 by Marei Peischl <marei@peitex.de>
%%
%% ============================================================================
%% This work may be distributed and/or modified under the
%% conditions of the LaTeX Project Public License, either version 1.3c
%% of this license or (at your option) any later version.
%% The latest version of this license is in
%% http://www.latex-project.org/lppl.txt
%% and version 1.3c or later is part of all distributions of LaTeX
%% version 2008/05/04 or later.
%%
%% This work has the LPPL maintenance status `maintained'.
%%
%% The Current Maintainers of this work are
%%   Marei Peischl <tuda-ci@peitex.de>
%%   Markus Lazanowski <latex@ce.tu-darmstadt.de>
%%
%% The development respository can be found at
%% https://github.com/tudace/tuda_latex_templates
%% Please use the issue tracker for feedback!
%%
%% If you need a compiled version of this document, have a look at
%% http://mirror.ctan.org/macros/latex/contrib/tuda-ci/doc
%% or at the documentation directory of this package (if installed)
%% <path to your LaTeX distribution>/doc/latex/tuda-ci
%% ============================================================================
%%
% !TeX program = lualatex
%%

\documentclass[
	ngerman,
	ruledheaders=section,%Ebene bis zu der die Überschriften mit Linien abgetrennt werden, vgl. DEMO-TUDaPub
	class=report,% Basisdokumentenklasse. Wählt die Korrespondierende KOMA-Script Klasse
	thesis={type=master},% Dokumententyp Thesis, für Dissertationen siehe die Demo-Datei DEMO-TUDaPhd
	accentcolor=9c,% Auswahl der Akzentfarbe
	custommargins=true,% Ränder werden mithilfe von typearea automatisch berechnet
	marginpar=false,% Kopfzeile und Fußzeile erstrecken sich nicht über die Randnotizspalte
	%BCOR=5mm,%Bindekorrektur, falls notwendig
	parskip=half-,%Absatzkennzeichnung durch Abstand vgl. KOMA-Script
	fontsize=11pt,%Basisschriftgröße laut Corporate Design ist mit 9pt häufig zu klein
%	logofile=example-image, %Falls die Logo Dateien nicht vorliegen
]{tudapub}

% MEINE PACKETE
\usepackage{syntax}

% Meine Packete
\usepackage{syntax}
\renewcommand{\syntleft}{}
\renewcommand{\syntright}{}
\usepackage{listings}
\def\lstlanguagefiles{lstlean.tex}

\usepackage{color}
\definecolor{keywordcolor}{rgb}{0.7, 0.1, 0.1}   % red
\definecolor{tacticcolor}{rgb}{0.0, 0.1, 0.6}    % blue
\definecolor{commentcolor}{rgb}{0.4, 0.4, 0.4}   % grey
\definecolor{symbolcolor}{rgb}{0.0, 0.1, 0.6}    % blue
\definecolor{sortcolor}{rgb}{0.1, 0.5, 0.1}      % green
\definecolor{attributecolor}{rgb}{0.7, 0.1, 0.1} % red

% Der folgende Block ist nur bei pdfTeX auf Versionen vor April 2018 notwendig
\usepackage{iftex}
\ifPDFTeX
	\usepackage[utf8]{inputenc}%kompatibilität mit TeX Versionen vor April 2018
\fi

%%%%%%%%%%%%%%%%%%%
%Sprachanpassung & Verbesserte Trennregeln
%%%%%%%%%%%%%%%%%%%
\usepackage[english, main=ngerman]{babel}
\usepackage[autostyle]{csquotes}% Anführungszeichen vereinfacht

% Falls mit pdflatex kompiliert wird, wird microtype automatisch geladen, in diesem Fall muss diese Zeile entfernt werden, und falls weiter Optionen hinzugefügt werden sollen, muss dies über
% \PassOptionsToPackage{Optionen}{microtype}
% vor \documentclass hinzugefügt werden.
\usepackage{microtype}

%%%%%%%%%%%%%%%%%%%
%Literaturverzeichnis
%%%%%%%%%%%%%%%%%%%
\usepackage{biblatex}   % Literaturverzeichnis
\bibliography{DEMO-TUDaBibliography}


%%%%%%%%%%%%%%%%%%%
%Paketvorschläge Tabellen
%%%%%%%%%%%%%%%%%%%
%\usepackage{array}     % Basispaket für Tabellenkonfiguration, wird von den folgenden automatisch geladen
\usepackage{tabularx}   % Tabellen, die sich automatisch der Breite anpassen
%\usepackage{longtable} % Mehrseitige Tabellen
%\usepackage{xltabular} % Mehrseitige Tabellen mit anpassbarer Breite
\usepackage{booktabs}   % Verbesserte Möglichkeiten für Tabellenlayout über horizontale Linien

%%%%%%%%%%%%%%%%%%%
%Paketvorschläge Mathematik
%%%%%%%%%%%%%%%%%%%
%\usepackage{mathtools} % erweiterte Fassung von amsmath
%\usepackage{amssymb}   % erweiterter Zeichensatz
%\usepackage{siunitx}   % Einheiten

%Formatierungen für Beispiele in diesem Dokument. Im Allgemeinen nicht notwendig!
\let\file\texttt
\let\code\texttt
\let\tbs\textbackslash
\let\pck\textsf
\let\cls\textsf

\usepackage{pifont}% Zapf-Dingbats Symbole
\newcommand*{\FeatureTrue}{\ding{52}}
\newcommand*{\FeatureFalse}{\ding{56}}

\begin{document}

\Metadata{
	title=Master Thesis WiSe2023,
	author=Simon Daniel
}

\title{Distributed Programming -- Funny things in Lean with session Types}
\subtitle{TODO Untertitel}
\author[S. Daniel]{Simon Daniel}%optionales Argument ist die Signatur,
\birthplace{Worms}%Geburtsort, bei Dissertationen zwingend notwendig
\reviewer{ 1. Prof. Dr.-Ing. Mira Mezini  \and 2. David Richter M.Sc.}%Gutachter

%Diese Felder werden untereinander auf der Titelseite platziert.
%\department ist eine notwendige Angabe, siehe auch dem Abschnitt `Abweichung von den Vorgaben für die Titelseite'
\department{inf} % Das Kürzel wird automatisch ersetzt und als Studienfach gewählt, siehe Liste der Kürzel im Dokument.
\group{Software Technology Group}

\submissiondate{\today}
\examdate{\today}

% Hinweis zur Lizenz:
% TUDa-CI verwendet momentan die Lizenz CC BY-NC-ND 2.0 DE als Voreinstellung.
% Die TU Darmstadt hat jedoch die Empfehlung von dieser auf die liberalere
% CC BY 4.0 geändert. Diese erlaubt eine Verwendung bearbeiteter Versionen und
% die kommerzielle Nutzung.
% TUDa-CI wird im nächsten größeren Release ebenfalls diese Anpassung vornehmen.
% Aus diesem Grund wird empfohlen die Lizenz manuell auszuwählen.
%\tuprints{urn=XXXXX,printid=XXXX,year=2022,license=cc-by-4.0}
% To see further information on the license option in English, remove the license= key and pay attention to the warning & help message.

% \dedication{Für alle, die \TeX{} nutzen.}

\maketitle

\affidavit
% Es gibt mit Version 3.20 die Möglichkeit ein Bild als Signatur einzubinden.
% TUDa-CI kann nicht garantieren, dass dies zulässig ist oder eine eigenhändige Unterschrift ersetzt.
% Dies ist durch Studierende vor der Verwendung abzuklären.
% Die Verwendung funktioniert so:
%\affidavit[signature-image={\includegraphics[width=\width,height=1cm]{example-image}}, <hier können andere Optionen zusätzlich stehen>]

\tableofcontents

\chapter*{Abstract}
\label{ch:abstract}
Parallel and distributed systems have been established for decades by now and the relevance of communication in the field of programming has steadily increased ever since. While such Software Systems could be written with traditional PL's and paradigms there arises a whole ne set of programming error possibilities. For example deadlocks, or sending unexpected types of data to a receiver.
To make programming DS more intuively and less error prone there has been high effort in creating specialized languages. those 'new' languages usally follow one of two paradigms.
a Choreography or a Multiparty view. A Choreography tries to describes the communication from some kind of global view like a play or a dance with a construct of a directed send that specifies both sender and receiver. Multiier programming on the other hand lets the programmer shift the location of computation in code and thus masking the required communication. Still it has been shown that both approaches share strong similiaritys (Multiparty Languages: The Choreographic and
Multitier Cases).
Session Types, first introduced 1998 (i guess earlier?) and standardized by W3C in 2002, are an handy tool that has been used to statically check multiparty programs. They describe the communication layout of two or more partys and can already make strong safety guarantees by cheap static analysis. There has been many new research regarding session Types since (2015 i guess nochmal nachsehen) that focused on extending session types and their expressiveness, \par but only a few cover session types in asychronous and unreliable environments. Fault tolrence in under these circumstances can be important, so i will investigate on the special challenges that arise here. Accompanying i implement a simple Choreography Model with session type projection in Lean4 that demonstrates some of my points

\chapter{Introduction}
\label{ch:introduction}
\section{research Questions}

Can a Choreographic Language be implemented in Lean4 in a similiar way to Haskell/HasChor? To Answer this Question i implement a DSL describing choreographic Programs in Lean.
\begin{itemize}
  \item Implement Knowledge of Choice mechanism
  \item Implement EPP Projection for Choreographic programs
  \item Implement 4 projection functions (Skizze)
  \item find a suitable computation model for choreos
  \item is the Type of a end point projected program \emph{always} equivalent to the end point projected type of the program?
\end{itemize}

%
% \subsection{challenges}
% \begin{itemize}
%   \item what changes when communication is unreliable (or just asychronous)? adapt my model in Lean accordingly to include failure
%   \item how to verify Choreographys supplied at runtime? im not quite sure if this is even neccessary after EPP
% \end{itemize}
% \subsection{notes}
% \begin{itemize}
%   \item unreliable nodes might be interesing? fault-tolerant and asynchronous systems were not covered by HasChor paper ``and other functional Choreographic languages'' do nothing to adress these difficulties (also mentioned in Modular Compilation for Higher-Order Functional Choreographies).
%   \item how can session times still yield safety guarantees in asynchronous and/or unreliable systems (Multiparty Asynchronous Session Types Honda.)
%   \item make Choreographys without obvious EPP projectable (like in Now it Compiles!). Are there other interesting problems than knowledge of choice?
%   \item might be interesting to look into runtime verification with session types. (havent red alot about this, more in the future work / outlook sections)
% \end{itemize}

\section{Main part, no sections in introduction i think}
This Theses proposes a libraray to formalize Choreographies inside the functional programming language Lean4 \cite{github_lean4}. It follows a recent idea inspired by the 2023 published paper \"HasChor: Functional Choreographic Programming for All\" while appliing the concept in and more recent programming language and extending the original implementation.
-- Outline
This Thesis is structured as follows: \cref{ch:showcase} will illustrate how the libraray can be used by following a simple example program. Then \cref{ch:backround} will set the stage and explain neccessary information for \cref{ch:implementation} where i present excerpts of my implementation, talk about the design choices and compare it against exisiting implementations.
In \cref{ch:evaluation} i talk about how well Choreographies integrate into the Lean programming language. I also present a few example programs to show the practical capabilities and compare them to ScalaLoci, Choral and Haschor.

\chapter{Choreography Showcase}
\label{ch:showcase}

\textbf{Choreography:}
\begin{lstlisting}[label={lst:buyer-seller-chor}, caption={Choreography}, language=lean, basicstyle=\tiny]
def book_seller (ep: Location): Choreo ep (Option (String @ buyer#ep)):= do

  let budget <- locally buyer get_budget
  let title <- locally buyer get_title
  let title' <- title ~> seller
  let price <- (lookup_price (⤉ title')) @ seller ~~> buyer
  let choice: Bool @ buyer <- locally buyer do return ((⤉budget) >= (⤉price))

  branch d fun
  | true => do
    let date <- deliveryDate @ seller ~~> buyer
    return some date
  | false => do
     return none
\end{lstlisting}
\noindent
\begin{minipage}{.5\textwidth}
\textbf{Buyer:}

\begin{lstlisting}[label={lst:buyer-ep}, caption={Buyer}, language=lean, basicstyle=\tiny]
def buyer: LocalM buyer (Option (String)):= do
  send seller title

  let price <- recv seller Nat
  let choice := (budget >= price)
  send seller choice
  if choice then
    let date <- recv seller String
    return some date
  else
    return none
\end{lstlisting}
\end{minipage}
\begin{minipage}{.5\textwidth}
\textbf{Seller:}
\begin{lstlisting}[label={lst:seller-ep}, caption={Buyer},language=lean, basicstyle=\small]
def seller: LocalM seller Unit:= do
  let title' <- recv buyer String
  let price <- (lookup_price  title')
  send buyer price
  let choice <- recv buyer Bool
  if choice then
    let date <- deliveryDate
    send buyer date
\end{lstlisting}
\end{minipage}


Here you can see the same behaviour implemented in two different ways. \cref{lst:test} shows the combined program specification, \cref{ch:evaluation} and \cref{ch:evaluation} show the individual program for the specified endpoints.
\par
The protocol includes two participants. A buyer who wants to order a book and a bookseller. To make the decision wether the \emph{buyer} can afford the Book, he first sends it's title to the \emph{seller}. The \emph{seller} is able to lookup the Books price in his local Database by the title and commuinicates the value back to the \emph{buyer}. Next The buyer checks if his budget fits the book price. If it does he requests the book from the Seller who finally notifies the buyer about the delivery date which terminates the protocol. If the book is to expensive the protocol terminates right away with a None Option.
This is a common Scenario to introduce Choreographys, for example \cite{pirouette} uses it, and \cite{gentle-introduction-multiparty-sessiontypes} uses the same protocol with different names.
Local programs \cref{lst:test} are written in the \emph{LocalM} Monad. This is Monad takes a Location as an argument where the local program is beeing executed and lets you accumulate Side effects specific to that Location. Notice in line 3+4 %TODO%
that \lstinline!get_budget! and \lstinline!get_title! are effects of the buyer location and \lstinline!lookup_price! in line 3 is a effect of the seller location. Addionally every LocalM monad can contain send and receive effects. Sending a Value takes a destination Location and a value to send as arguments, receiving a value takes the source and the expected Type %TODO
Having these two programs specified as two seperate functions opposes several different problems. Lets say the protocol requirements change and the book buyer additionally sends an ISBN to the Title to better identify the book. If you just change the buyer program specification you end up in a deadlock, since the buyer wil block on the send of the ISBN and the seller will block on its send of the book price. Both parties arent able to progress further now. A even worse scenario might be if one participant for example changes the type of the Delivery date, lets say from String to Int. The buyer would expect the wrong datatype and interpret the received data false, leading to junk values that a user might not even notice.
One Solution to this problem is to combine both programs into one. My libraray provides the Choreo ep monad which lets you combine the buyer and seller programs into one bookseller Choreography. Here you can execute local programs with the \emph{locally} function and suppliing the Location for the program to run on. Line 2 in TODO corresponds to line 2 in TODO. However the returned budget is not a normal Nat value anymore, it is Nat value that only lives on the buyer Location.%TODO
Values can change Locations in one single operation instead of seperate send and receive calls. This operation in line 4 is written by the notation \emph{~>}. This translates to: send title from the location it lives on, namely the buyer, to the seller location. Line 5 shows a short notation for executing a local program and sending the result right away, the part before the @ is the local Program, he part after is the program location and part after the ~~> arrow is the receiving locaiton.

\chapter{Background}
\label{ch:background}
\section{Dictionary}

(clarify the use of Agent, nodes, locations and be consistent throughout the thesis.)
Literature uses different kinds of naming conventions for talking about communication paricipants, sometimes to express a slightly different intent that share semantically the same properties. I will stick to location throughout my work, in other work you might read terms like nodes, parties, Agents. The paper also presents several extensions (... maybe add whats important)

\subsection{deadlocks}
A deadlock occurs when a computer prgoram ist not able to progress any further. This is usally considered as undesired behaviour, even if the program was not able to finish its task successfully there still remain unused processes on the CPU that might not be able to release aquired resources any more to the System. On large scale Distributed Systems this can lead to potentially large expenses for the hardware operator. An Textbook example would be set up by 2 Processes both performing a blocking send operation other the network but never being able to reach the complementary receive.
\subsection{Choreography}
A Choreography is program specification with first party support for distributed computation. Usual imperative or functional PLs target a single CPU by default to run on, meaning the compiler will only output a single executable prgoram file. Approaches to Multiparty programming might be to branch inside the Programm depending on the executing Node/location ID or write different programs from the ground up.
By using a common choreographic specification, however, programmers gain a few crucial benefits. The System behaviour can be written in one common place, the intent of and order of communication can be expressed more intuively and it can be easier to argue about safety guarantees as we see later on.

\subsection{higher order Choreography}
Higher order Choreos are Choreos that can take other Choreos as an Argument. This can be very handy and be thought of subprotocols. An example would be a Choreo where you act as a buyer that wants to buy books from a seller. Additionaly you have a friend and are negotiating with him how much money he wants to lend you. The negotiation choreo between you and your friend can now be taken as an argument for the book buying Choreo. The benefit is that details of the negotiation arent neccessary anymore in a static context, but can be supplied at runtime!

(wenn ich es habe kann ich hier ja mein Beispiel rein bringen)


\subsection{Knowledge of Choice}

Whenever a party inside a Choreography branches on some kind of runtime condition other parties might have to be informed of that branching choice. this does not allways have to be the case, but only if a second party's communication pattern depends on the branch decision it has to aquire the 'Knowledge of Choice' through a notification of some kind.

\subsection{Session Types}
or Program Types describe the communication patterns and can be thought of as protocols. In my Paper I differntiate between global and local (session) Types. By Global i mean the global view over \emph{all} locations and local being the view of a single actor/location. The Global Type describes the protocol of the Multiparty program, always specifiing sender and receiver for communication operations. In contrast a local type only cares about one side, as for send operations the sender is implicetely set to the location of the local type and vice versa for receiving. In a Multi Party prgoram with more than 2 participants, a local type may ignore all communication between locations that does not include itself.

globalType ->
\newline | a -> b: dataType. globalType
\newline | end

dataType ->
string
\newline | number
\newline | globalType

\section{Sessions and Session Types:
an Overview}
Session Types take an important role for Choreos and describe the interaction of two (binary) or more (multi party) locations at communication level. We differentiate between \textbf{Global Session Types} that describe the sequence of directed message exchange for a session, and \textbf{Local Session Types}, which reflect the same Protocol but from one locations's perspective. The projection from a global session type to a local ST is called \textbf{End Point Projection} and basicallay translates all sending operations to send operations if location being projected to is sending, or receive operations if is receiving.

\section{MultiParty Session Types}


Essential Session Type Syntax introduced by K. Honda in the 1990

\begin{grammar}

<T> ::= end
	\alt !S.S' -- Send
	\alt ?S.S' -- Receive

\end{grammar}

\begin{grammar}
<G> ::= end
	\alt $\mu t. G$
	\alt t
	\alt $ p \rightarrow q: \{l_i(S_i).G_i\}_{i \in I}$

\end{grammar}


Syntax of Introduction to MPST

\begin{grammar}

<S> ::= nat
	\alt int
	\alt bool

\end{grammar}

\begin{grammar}

<T> ::= end
	\alt $ \&_{i \in I} p?l_i(S_i).T_i$
	\alt $\oplus _{i \in I} q!l_i(S_i).T_i$
	\alt $\mu t.T$
	\alt t

\end{grammar}

Syntax of Mira / TU Darmstadt
extends the previous Global Type Syntax by the following:
\begin{grammar}

<G> ::= [b]G
	\alt G|G
	\alt G+G
	\alt $\prod x:I.G$
	\alt G e

\end{grammar}

Extension to Local Types
\begin{grammar}

<T> ::= [b]T
	\alt T+T
	\alt $\prod x:I.T$
	\alt T e

\end{grammar}

Global Type Implementation in Lean4

% \begin{lstlisting}[language=lean]
% inductive P where
%   | IF 			: located Exp -> P -> P -> P
%   | SEND_RECV   : located Exp -> located Variable -> P -> P
%   | COMPUTE (v: Variable) (e: Exp) (a: Location) :   P -> P
%   | END     	: Exp -> Location -> P
% \end{lstlisting}

\section{HasChor: Functional Choreographic Programming for All
(Functional Pearl)}
this paper is about Choreographic programming and an implementation for the Haskell Language. Choreographies are implemented as a Library for use with Haskell and expressed as Computational monads. Might be helpful for own Implementations in Lean to look at their Design Choices for Language Design. Special remarks are on the \textbf{higher order} capabilities, meaning Choreographies can take other Choreos as a parameter as some kind of sub-protocol, as well as \textbf{location polymorphism}. Location polymorphism here means the runtime substitution of locations.

\section{Modular Compilation for Higher-Order Functional Choreographies}
Higher order Choreographies make it non trivial to check which locations are involved in the Choreography. For example if Choreo C depends on Choreo K, the participants of C are only known after K is instantiated. This poses problems for static checks and the EPP. The paper proposes approaches to extend the lambda calcus to achieve a modular applications of an EPP (did not fully understand alot)

\section{On the Monitorability of Session Types, in Theory
and Practice}
about runtime checking and monitoring of communication but very long

\section{Multiparty Languages: The Choreographic and
Multitier Cases}

this work presents the two programming paradigms Choreographic programming and Multitier. Multitier being programm descriptions that specify the location of operations instead of explicit communication. While both approaches have different roots and went through different development, mainly because both workgroups rarely mix or push collective results, there still is a fundamental linkage between both. This similiaritys could lead both camps to \textbf{cross-fertilisation} that benefits both. This statement is backed by the comparison of two stripped down versions of an Choreo PL and a Multitier one.

\section{Certified Automatic Repair of Uncompilable Protocols}
There are Choreos that are not (directly) projectable to local Programs / types. Branching adds the challenges for the session, since continuation of a location protocal might depend on some choice that leads to different executions of two or more sub Protocols. This issue can be fixed by informing other Party members of branching choices that a location might does, called \textbf{knowledge of choice}. The paper also describes more clever ways of propagating choices, since not all choices are relevant to all locations and choices can be transitively propageted for example. The Automated Process is called \textbf{Amendment} or \textbf{repair}.


\section{Mathlib}
My implementation also depends on small parts of the mathlib4 Library. Mathlib4 is a community-driven project to build a big collection of mathematics formalized in Lean4. It is the most popular Lean4 library by far (673 Stars on github) and besides the mathimatical emphasis it also provides useful definitions for programming \cite{leanprovercommunity}. In particular i used the library for the dlookup function Lists of dependent Types (Mathlib.Data.List.Sigma) and the Finenum Typeclass (Mathlib.Data.FinEnum).
Even though Mathlib is a big dependency i think it pays of with its features. It also extends the Lean standard Library for example by additional theorems, also features that are mature enough and seem useful to the community get shifted into the Lean standard. It is a similiar proving ground like Boost is for c++.

\chapter{Implementation}
\label{ch:implementation}


\section{Prgram Syntax}

Choreo (Multiparty Program) P -> if E then P else P endif P
\newline| send_receive 		:v@l <= v@l2
\newline| compute      		: v@l <= E@l
\newline| choreo 			: v@l <= CHOREO with mapping P END_CHOREO P
\newline| end 				: v@l

Expression E -> Variable: v
\newline| CONSTANT : c
\newline| PLUS     : (E+E)
\newline| DIVIDE   : (E/E)
\newline| Function -> f(E)

v sind Variablennamen (Strings) und l Location Namen (Strings)

\begin{grammar}
<Choreo> ::= <ident> ‘=’ <expr>
\alt ‘for’ <ident> ‘=’ <expr> ‘to’ <expr> ‘do’ <statement>
\alt ‘{’ <stat-list> ‘}’
\alt <empty>
<stat-list> ::= <statement> ‘;’ <stat-list> | <statement>
\end{grammar}


\begin{grammar}
<Choreo> ::= <ident> ‘=’ <expr>
\alt ‘for’ <ident> ‘=’ <expr> ‘to’ <expr> ‘do’ <statement>
\alt ‘{’ <stat-list> ‘}’
\alt <empty>
<stat-list> ::= <statement> ‘;’ <stat-list> | <statement>
\end{grammar}

\section{Networking in Lean4}
Lean4's standard Library provides IO operations for exchanging values between multiple Threads. While this would suffice to test the functionality of running Choreographies on a single machine, it is much more desirable to be able to split the Choreography into actual seperate programs that may run on different machines. To accomplish this I decided to implement message exchanging by Sockets.
While some PLs like Java provide a Language embedded implementation of Sockets, or other well established Languages like Haskell provide mature Librarays for a Socket API, Lean has a much younger ecosystem that seems less feature rich. The Solution i chose is the relatively young Lean4 library 'sockets for Lean 4' by Henrik Böving (TODO in Quellen reinpacken) which is hosted at https://github.com/hargoniX/socket.lean under the MIT License. It servers as a wrapper around the platform specific 'winsock.h' Windows, and the 'sys/socket.h' Unix C-librarys while providing a clean Lean interface.
\par
To abstract away the Socket setup code in my local programs i decided to reopen a new socket for every communication operation in the Choreography for simplicity

\begin{lstlisting}[language=lean]
def address.send (a: address) (msg: t) [Serialize t]: IO Unit := do
  let bytes := Serialize.to_bytes msg
  let sock ← Socket.mk .inet .stream
  sock.connect a
  let sz ← sock.send bytes
  sock.close
\end{lstlisting}

similiarly to the sock.connect operation, the 'receive' function will bind an listen to a socket with the corresponding addres. For convenience a 'broadcast' function is also available, that takes an 'List address' instead of a single address and sends msg to all addresses in the List.
\newline
Serialize is a simple TypeClass that provides the following two functions:
\begin{lstlisting}[language=lean]
class Serialize (a: Type) extends ToString a where
  to_bytes: a -> ByteArray
  from_bytes: ByteArray -> Except String a
\end{lstlisting}
By implementing this TypeClass a Type can converted to, and from a ByteArray, enabling it to be sent over the network.

\section{Distributed Values}
When computing with values in a global choreographic description it is crucial to differtiate between ``normal'' and distributed values. Since the choreography projects to different executable programs, distributed values can only live at a single location at a time. The authors of HasChor used Language extensions to integrate the location of a value into the type system. This is important to statically distinguish between a Value that lives on a location l1 and a value on l2 where l1 != l2.
Lean's dependent Typesystem however provides more flexibility and lets express the same dependent type without any language extensions or third party librarys like follows:
\begin{lstlisting}[language=lean]
inductive LocVal (α: Type) (loc: String) where
| Wrap: α -> LocVal α loc
| Empty: LocVal α loc
\end{lstlisting}
Here \emph{LocVal.Wrap} elements are representing Values at the Location \emph{loc} where LocVal simply holds a Value of a given type \emph{ alpha }.
The \emph{Empty} Option represents the abscence of an actual \emph{alpha} Value for all locations other than \emph{loc}. This is important in the endpoint projection process. We present two functions to go from and to a LocVal from a given Value of any Type alpha:

\begin{lstlisting}[language=lean]
def wrap {a} (v:a) (l: String): a @ l:=
  LocVal.Wrap v

def unwrap (lv: a @ l) (ex: notEmpty lv):  a := match lv with
| LocVal.Wrap v =>  v

\end{lstlisting}

 

\chapter{Conclusion}
\label{ch:conclusion}


\printbibliography

\end{document}
