\chapter{Introduction}
\label{ch:introduction}
\section{research Questions}

Can a Choreographic Language be implemented in Lean4 in a similiar way to Haskell/HasChor? To Answer this Question i implement a DSL describing choreographic Programs in Lean.
\begin{itemize}
  \item Implement Knowledge of Choice mechanism
  \item Implement EPP Projection for Choreographic programs
  \item Implement 4 projection functions (Skizze)
  \item find a suitable computation model for choreos
  \item is the Type of a end point projected program \emph{always} equivalent to the end point projected type of the program?
\end{itemize}


\subsection{challenges}
\begin{itemize}
  \item what changes when communication is unreliable (or just asychronous)? adapt my model in Lean accordingly to include failure
  \item how to verify Choreographys supplied at runtime? im not quite sure if this is even neccessary after EPP
\end{itemize}
\subsection{notes}
\begin{itemize}
  \item unreliable nodes might be interesing? fault-tolerant and asynchronous systems were not covered by HasChor paper ``and other functional Choreographic languages'' do nothing to adress these difficulties (also mentioned in Modular Compilation for Higher-Order Functional Choreographies).
  \item how can session times still yield safety guarantees in asynchronous and/or unreliable systems (Multiparty Asynchronous Session Types Honda.)
  \item make Choreographys without obvious EPP projectable (like in Now it Compiles!). Are there other interesting problems than knowledge of choice?
  \item might be interesting to look into runtime verification with session types. (havent red alot about this, more in the future work / outlook sections)
\end{itemize}
