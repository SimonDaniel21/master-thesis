\chapter{Examples}

This Section presents four choreographic programs, to demonstrate how choreographies can be written in Lean using the chorlean library. The first example \ref{ss:silent} shows a very basic communication scheme. In the \ref{ss:book} example a already more complex protocol is presented to highlight higher Level Choreographies aswell as Location Polymorphism. \ref{ss:sso} Shows an application of a simple cryptographic protocol for a SSO service and the last example \ref{ss:merge} showcases a compute heavy application that balances compute load of a sorting algorithm across a set of  nodes.

\subsection{silent post}
\label{ss:silent}
This first Distributed protocol is kept simple to demonstrate some of the features(TODO einen richtigen Satz formulieren).
Three Participants, Alice, bob and eve take part in this protocol. The Scheme is that alice prompts an text input from the user, Then passes this input arround in a circle through bob and eve. Both bob and eve perfom local modifications of the message in the spirit of the silent post game.
%\input{chapters/examples/silentpost}
As the Type of a Choreo depends on the endpoint it is projected to, the silent_post function carries an endpoint as a argument. Also notice the \emph{@} notation, that is an abbreviation for an \emph{GVal} Type with implicit endpoint. This notation works inside a Choreo as the endpoint of GVal's resulting from a communication or local operation is allways the endpoint the choreography is parameterized with.
As discussed in --TODO-- the locally functions lets uns define a Freer-Monad with all locally available effects. For Silent post, those local Effect Signatures are defined as
\begin{lstlisting}[language=lean]
instance sig: LocSig Location where
  sig x := match x with
    | alice =>  CmdInputEff ⨳ LogEff
    | bob =>  LogEff
    | eve => LogEff
\end{lstlisting}
such that every location is allowed to perform a logging effect \emph{LogEff} and additionally alice has access to CmdInputEff, which allow to prompt a String from a User in the command line.


\subsection{book seller}
\label{ss:book}
\begin{lstlisting}[language=lean]
def book_seller (negotiate: negT buyer ep)
  : Choreo ep (Option (String @ buyer # ep)) := do

  let budget <- locally buyer do get_budget
  let title <- locally buyer do get_title

  let title' <- (title ~> seller)
  let price <- locally seller do lookup_price (⤉title')
  let price <- price ~> buyer

  locally seller do info s!"got book title: {⤉title'}"

  locally buyer do info s!"the price is {⤉price}, negotiate with friend"

  let d <- negotiate budget price -- calls another choreo :)

  branch d fun
  | true => do
    let date <- locally seller do deliveryDate
    let date <- date ~> buyer
    return some date
  | false => do
    locally seller do warning s!"the customer declined the purchase"
    locally buyer do error s!"{⤉title} has a price of {⤉price} exceeding your budget of {⤉budget}!"
    return none

\end{lstlisting}


\subsection{sso authentication}
\label{ss:sso}

\begin{lstlisting}[language=lean]
structure Credentials where
  username: String
  password: String

def authenticate (ep:Location) (creds: Credentials @ client # ep):
  Choreo ep (Option ((String @ client # ep) × (String @ service # ep))):= do

  let pw <- locally client do return (⤉creds).password
  locally service do info "hello service"
  let username <- locally client do return (⤉creds).username
  let username' <- username ~> IP
  let salt <- locally IP do return add_salt (⤉username')
  let salt <- salt ~> client
  let hash <- locally client do return (calcHash (⤉salt) (⤉pw))
  let hash <- hash ~> IP
  let valid <- locally IP do check_hash (⤉hash)

  branch valid fun
  | true => do
    let token <- locally IP do IPEff.createToken
    let token_c <- token ~> client
    let token_s <- token ~> service
    return (token_c, token_s)
  | false =>
    return none
\end{lstlisting}

\subsection{mergesort}
\label{ss:merge}

%\begin{lstlisting}[language=lean]
def silent_post (ep:Location): Choreo ep Unit:= do
  let input <- locally alice do
    info "enter a message"
    return <- readString

  let msg: String @ bob <- input ~> bob
  let msg <- locally bob do return [(⤉msg), "bob"]

  let msg <- msg ~> eve
  let msg <- locally eve do return (⤉msg).concat "eve"

  let msg <- send_recv msg alice
  locally alice do info s!"finished with string from eve: {msg}"
\end{lstlisting}

