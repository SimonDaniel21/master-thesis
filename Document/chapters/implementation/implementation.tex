\chapter{Implementation}
\label{ch:implementation}

this chapter presents the Lean Code of my library implementation in detail. It also compares it to different implementations and discusses the design choices being made.
\par
throughout the code im using the following variables for the respective context.
here is a table of symbols and how i use them in this chapters code

$\mu$ : Type that can be transfered over the network, has an Serialize instance
$\delta$: Choreos are parameterized by the Type of locations called $\delta$
$\alpha$,$\beta$: Types
ep: short for 'endpoint'. This is a value of type $\delta$ that represents the location a local program is meant to be run on

Lean allows you to specify \"variables\", not to be confused with global variables as in global state from other PLs.
\begin{lstlisting}[language=lean]
variable {μ: Type} [Serialize μ]
def f (v:μ) := Serialize.to_bytes v
\end{lstlisting}

Here, even though $\mu$ does not provide a Serialize instance explictely but it is implicetely inserted through the variable statement. This allows to omit the Serialize instances in my implementation code.
%\section{Prgram Syntax}
%
% Choreo (Multiparty Program) P -> if E then P else P endif P
% \newline| send_receive 		:v@l <= v@l2
% \newline| compute      		: v@l <= E@l
% \newline| choreo 			: v@l <= CHOREO with mapping P END_CHOREO P
% \newline| end 				: v@l
%
% Expression E -> Variable: v
% \newline| CONSTANT : c
% \newline| PLUS     : (E+E)
% \newline| DIVIDE   : (E/E)
% \newline| Function -> f(E)
%
% v sind Variablennamen (Strings) und l Location Namen (Strings)
%
% \begin{grammar}
% <Choreo> ::= <ident> ‘=’ <expr>
% \alt ‘for’ <ident> ‘=’ <expr> ‘to’ <expr> ‘do’ <statement>
% \alt ‘{’ <stat-list> ‘}’
% \alt <empty>
% <stat-list> ::= <statement> ‘;’ <stat-list> | <statement>
% \end{grammar}
%
%
% \begin{grammar}
% <Choreo> ::= <ident> ‘=’ <expr>
% \alt ‘for’ <ident> ‘=’ <expr> ‘to’ <expr> ‘do’ <statement>
% \alt ‘{’ <stat-list> ‘}’
% \alt <empty>
% <stat-list> ::= <statement> ‘;’ <stat-list> | <statement>
% \end{grammar}

\section{Networking in Lean4}
Lean4's standard Library provides IO operations for exchanging values between multiple Threads. While this would suffice to test the functionality of running Choreographies on a single machine, it is much more desirable to be able to split the Choreography into actual seperate programs that may run on different machines. To accomplish this I decided to implement message exchanging by Sockets.
While some PLs like Java provide a Language embedded implementation of Sockets, or other well established Languages like Haskell provide mature Librarays for a Socket API, Lean has a much younger ecosystem that seems less feature rich. The Solution i chose is the relatively new Lean4 library 'sockets for Lean 4' by Henrik Böving (TODO in Quellen reinpacken) which is hosted at https://github.com/hargoniX/socket.lean under the MIT License. It servers as a wrapper around the platform specific 'winsock.h' Windows, and the 'sys/socket.h' Unix C-librarys while providing a native Lean interface. At the time of working on this thesis the \emph{Sockets} did not work with the, at that time, newest Version of Lean 4.6.0. Thanks to them they updated an underlying library (https://github.com/tydeu/lean4-alloy) and accepted my Merge Request with the version bump \cite{test}.
\par
To abstract away the Socket setup code in my local programs i decided to reopen a new socket for every communication operation in the Choreography for simplicity

\begin{lstlisting}[language=lean]
def address.send (a: address) (msg: t) [Serialize t]: IO Unit := do
  let bytes := Serialize.to_bytes msg
  let sock ← Socket.mk .inet .stream
  sock.connect a
  let sz ← sock.send bytes
  sock.close
\end{lstlisting}


similiarly to the sock.connect operation, the 'receive' function will bind an listen to a socket with the corresponding addres. For convenience a 'broadcast' function is also available, that takes an 'List address' instead of a single address and sends msg to all addresses in the List.

Serialize is a simple TypeClass that provides the following two functions:
\begin{lstlisting}[language=lean]
class Serialize (μ: Type) μ where
  to_bytes: μ -> ByteArray
  from_bytes: ByteArray -> Except String μ
\end{lstlisting}

By implementing the Serialize class, a Type can converted to, and from a ByteArray, enabling it to be sent over the network. Note that \lstinline!to_bytes! is a partial function that may not return a conversion of that type. Not every ByteArray must be convertible into a type. A ByteArray created by the to_bytes function however should always be able to convert back into the initial value by the from_bytes function. Lean lets you add this condition as an additional class Field like
\begin{lstlisting}[language=lean]
 correct: ∀ (v: μ), (from_bytes (to_bytes v)) = v
\end{lstlisting}

(with an addtional assumption that from_bytes is no Exception). I omitted this proof in my implementation since writing such rigorous proofs for Serialize instances like Lists would be out of scope of this thesis, as im not trying to implement a perfect Serialization Framework for Lean.

\subsection{Network Effects}
\begin{lstlisting}[language=lean]
inductive NetEff (ep: δ): Type -> Type 1
| send {μ: Type} [Serialize μ] : (r:δ) -> (r ≠ ep) -> μ -> NetEff ep Unit
| recv : (r:δ) -> (r ≠ ep) ->  (μ: Type) -> [Serialize μ] -> NetEff ep μ
\end{lstlisting}


\subsection{Persitant Connections}

\begin{lstlisting}[language=lean]
structure SockChannel (sender receiver ep: δ) where
  recv_sock: GVal receiver ep Socket
  send_sock: GVal sender ep Socket
\end{lstlisting}


\section{Locations}

explain the chosen Type for location and why it differs from HasChor, pros cons

\section{Distributed Values}
When computing with values in a global choreographic description it is crucial to differtiate between ``normal'' and distributed values. Since the choreography projects to different executable programs, distributed values can only live at a single location at a time. The authors of HasChor used Language extensions to integrate the location of a value into the type system. This is important to statically distinguish between a Value that lives on a location l1 and a value on l2 where l1 != l2.
Lean's dependent Typesystem however provides more flexibility and lets express the same dependent type without any language extensions or third party librarys like follows:

\begin{lstlisting}[language=lean]
inductive GVal (owner ep: δ) (α: Type) where
| Wrap:  (owner = ep) -> α -> GVal owner ep α
| Empty: (owner ≠ ep) -> GVal owner ep α
\end{lstlisting}

Here \emph{LocVal.Wrap} elements are representing Values at the Location \emph{loc} where LocVal simply holds a Value of a given type \emph{ alpha }.
The \emph{Empty} Option represents the abscence of an actual \emph{alpha} Value for all locations other than \emph{loc}. This is important in the endpoint projection process. We present two functions to go from and to a LocVal from a given Value of any Type alpha:

\begin{lstlisting}[language=lean]
inductive LocVal (α: Type) (loc: String) where
| Wrap: α -> LocVal α loc
| Empty: LocVal α loc
\end{lstlisting}



\section{Effects and Free Monads}
In this section i will describe the effect System that i implemented in Lean
\begin{lstlisting}[language=lean]
inductive Freer (Eff:Type u -> Type v) (α:Type u) where
| Do: Eff β -> (β -> Freer Eff α) -> Freer Eff α
| Return: α -> Freer Eff α
\end{lstlisting}

\begin{lstlisting}[language=lean]
inductive EmptyEff: Type → Type 1
\end{lstlisting}

\begin{lstlisting}[language=lean]
\end{lstlisting}

